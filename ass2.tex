%----------------------------------------------------------------------------------------
%	PACKAGES AND OTHER DOCUMENT CONFIGURATIONS
%----------------------------------------------------------------------------------------

\documentclass{article}

\usepackage{fancyhdr} % Required for custom headers
\usepackage{lastpage} % Required to determine the last page for the footer
\usepackage{extramarks} % Required for headers and footers
\usepackage[usenames,dvipsnames]{color} % Required for custom colors
\usepackage{graphicx} % Required to insert images
\usepackage{listings} % Required for insertion of code
\usepackage{courier} % Required for the courier font
\usepackage{amsmath,amsfonts,amsthm} % Math packages
\usepackage{enumitem}
% Margins
\topmargin=-0.45in
\evensidemargin=0in
\oddsidemargin=0in
\textwidth=6.5in
\textheight=9.0in
\headsep=0.25in

\linespread{1.1} % Line spacing

% Set up the header and footer
\pagestyle{fancy}
\lhead{\hmwkAuthorName} % Top left header
\rhead{\firstxmark} % Top right header
\lfoot{\lastxmark} % Bottom left footer
\cfoot{} % Bottom center footer
\rfoot{Page\ \thepage\ of\ 6} % Bottom right footer
\renewcommand\headrulewidth{0.4pt} % Size of the header rule
\renewcommand\footrulewidth{0.4pt} % Size of the footer rule

\setlength\parindent{0pt} % Removes all indentation from paragraphs
\providecommand{\abs}[1]{\lvert#1\rvert}

%----------------------------------------------------------------------------------------
%	DOCUMENT STRUCTURE COMMANDS
%	Skip this unless you know what you're doing
%----------------------------------------------------------------------------------------

% Header and footer for when a page split occurs within a problem environment
\newcommand{\enterProblemHeader}[1]{
	\nobreak\extramarks{#1}{#1 continued on next page\ldots}\nobreak
	\nobreak\extramarks{#1 (continued)}{#1 continued on next page\ldots}\nobreak
}

% Header and footer for when a page split occurs between problem environments
\newcommand{\exitProblemHeader}[1]{
	\nobreak\extramarks{#1 (continued)}{#1 continued on next page\ldots}\nobreak
	\nobreak\extramarks{#1}{}\nobreak
}

\setcounter{secnumdepth}{0} % Removes default section numbers
\newcounter{homeworkProblemCounter} % Creates a counter to keep track of the number of problems

\newcommand{\homeworkProblemName}{}
\newenvironment{homeworkProblem}[1][Question \arabic{homeworkProblemCounter}]{ % Makes a new environment called homeworkProblem which takes 1 argument (custom name) but the default is "Problem #"
	\stepcounter{homeworkProblemCounter} % Increase counter for number of problems
	\renewcommand{\homeworkProblemName}{#1} % Assign \homeworkProblemName the name of the problem
	\section{\homeworkProblemName} % Make a section in the document with the custom problem count
	\enterProblemHeader{\homeworkProblemName} % Header and footer within the environment
}{
	\exitProblemHeader{\homeworkProblemName} % Header and footer after the environment
}

\newcommand{\problemAnswer}[1]{ % Defines the problem answer command with the content as the only argument
	\noindent\framebox[\columnwidth][c]{\begin{minipage}{0.98\columnwidth}#1\end{minipage}} % Makes the box around the problem answer and puts the content inside
}

\newcommand{\homeworkSectionName}{}
\newenvironment{homeworkSection}[1]{ % New environment for sections within homework problems, takes 1 argument - the name of the section
	\renewcommand{\homeworkSectionName}{#1} % Assign \homeworkSectionName to the name of the section from the environment argument
	\subsection{\homeworkSectionName} % Make a subsection with the custom name of the subsection
	\enterProblemHeader{\homeworkProblemName\ [\homeworkSectionName]} % Header and footer within the environment
}{
	\enterProblemHeader{\homeworkProblemName} % Header and footer after the environment
}

%----------------------------------------------------------------------------------------
%	NAME AND CLASS SECTION
%----------------------------------------------------------------------------------------

\newcommand{\hmwkTitle}{Project\ \#3} % Assignment title
\newcommand{\hmwkDueDate}{Wednsday,\ 4\ Apr 2014} % Due date
\newcommand{\hmwkClass}{MA3237\ } % Course/class
\newcommand{\hmwkAuthorName}{Chen Yu}
\newcommand{\MatricNumber}{A0077976E}
\newcommand{\AuthorDescription}{Engineering Science Programme}

%----------------------------------------------------------------------------------------
%	TITLE PAGE
%----------------------------------------------------------------------------------------

\title{
	\vspace{2in}
	\textmd{\textbf{\hmwkClass:\ \hmwkTitle}}\\
	\normalsize\vspace{0.1in}\small{Due\ on\ \hmwkDueDate}
	\vspace{3in}
}

\author{
	\textbf{\hmwkAuthorName}\\
	\textbf{\MatricNumber}\\
	\textbf{\AuthorDescription}
}
\date{} % Insert date here if you want it to appear below your name

%----------------------------------------------------------------------------------------

\begin{document}

\maketitle


%----------------------------------------------------------------------------------------
%	PROBLEM 1
%----------------------------------------------------------------------------------------

% To have just one problem per page, simply put a \clearpage after each problem

\newpage
\begin{homeworkProblem}
	\section{Code} 
	\begin{lstlisting}[language=Matlab]
clear;
close all;
function I = repI(i)
	I = [];
	for j = 1:i
		I = [I,'I'];
	end
end

nTrials=2.^(1:20); m=length(nTrials);

vEst = cell(3,1);
for i = 1:3
	vEst{i} = zeros(1,m);
end

rand('state',0);
for i=1:m
    n=nTrials(i);
    x = rand(1,n);
    vEst{1}(i) = sum(4*sqrt(1-x.^2))/n;
    x = -sqrt(4 - 3*rand(1,n)) + 2;
    vEst{2}(i) = sum(12*sqrt(1-x.^2)./(4-2*x))/n;
    x = -sqrt(1 - rand(1,n)) + 1;
    vEst{3}(i) = sum(4*sqrt(1-x.^2)./(2-2*x))/n;
end

err = cell(3,1);
for i = 1:3
	err{i} = vEst{i} - pi;
end

equation = cell(3,1);
equation{1} = '1$';
equation{2} = '\frac{4-2x}{3}$';
equation{3} = '2-2x$';

for i = 1:3
	the_plot = figure(i);
	subplot(2,1,1);
	semilogx(nTrials,vEst{i});
	title(['Method ',repI(i),': $g(x) =', equation{i}]);

	subplot(2,1,2);
	loglog(nTrials,abs(err{i}));
	hold on;
	loglog(nTrials,1./sqrt(nTrials),'r:'); title(['Method ', repI(i)])
	legend('$\abs{err}$ vs number of trials','$\frac{1}{\sqrt{number of trials}}$');
	print(the_plot,['Method_',repI(i),'.tex'],'-S500,450','-dtex')
end
	\end{lstlisting}  

	\section{Figures}
	\begingroup
		% Title: glps_renderer figure
% Creator: GL2PS 1.3.8, (C) 1999-2012 C. Geuzaine
% For: Octave
% CreationDate: Thu May  1 14:38:04 2014
\setlength{\unitlength}{1pt}
\begin{picture}(0,0)
\includegraphics{Method_I-inc}
\end{picture}%
\begin{picture}(500,450)(0,0)
\fontsize{10}{0}
\selectfont\put(65,257.722){\makebox(0,0)[t]{\textcolor[rgb]{0,0,0}{{1e+0}}}}
\fontsize{10}{0}
\selectfont\put(120.357,257.722){\makebox(0,0)[t]{\textcolor[rgb]{0,0,0}{{1e+1}}}}
\fontsize{10}{0}
\selectfont\put(175.714,257.722){\makebox(0,0)[t]{\textcolor[rgb]{0,0,0}{{1e+2}}}}
\fontsize{10}{0}
\selectfont\put(231.071,257.722){\makebox(0,0)[t]{\textcolor[rgb]{0,0,0}{{1e+3}}}}
\fontsize{10}{0}
\selectfont\put(286.429,257.722){\makebox(0,0)[t]{\textcolor[rgb]{0,0,0}{{1e+4}}}}
\fontsize{10}{0}
\selectfont\put(341.786,257.722){\makebox(0,0)[t]{\textcolor[rgb]{0,0,0}{{1e+5}}}}
\fontsize{10}{0}
\selectfont\put(397.143,257.722){\makebox(0,0)[t]{\textcolor[rgb]{0,0,0}{{1e+6}}}}
\fontsize{10}{0}
\selectfont\put(452.5,257.722){\makebox(0,0)[t]{\textcolor[rgb]{0,0,0}{{1e+7}}}}
\fontsize{10}{0}
\selectfont\put(60.0065,262.727){\makebox(0,0)[r]{\textcolor[rgb]{0,0,0}{{2.2}}}}
\fontsize{10}{0}
\selectfont\put(60.0065,286.082){\makebox(0,0)[r]{\textcolor[rgb]{0,0,0}{{2.4}}}}
\fontsize{10}{0}
\selectfont\put(60.0065,309.437){\makebox(0,0)[r]{\textcolor[rgb]{0,0,0}{{2.6}}}}
\fontsize{10}{0}
\selectfont\put(60.0065,332.792){\makebox(0,0)[r]{\textcolor[rgb]{0,0,0}{{2.8}}}}
\fontsize{10}{0}
\selectfont\put(60.0065,356.147){\makebox(0,0)[r]{\textcolor[rgb]{0,0,0}{{3}}}}
\fontsize{10}{0}
\selectfont\put(60.0065,379.502){\makebox(0,0)[r]{\textcolor[rgb]{0,0,0}{{3.2}}}}
\fontsize{10}{0}
\selectfont\put(60.0065,402.857){\makebox(0,0)[r]{\textcolor[rgb]{0,0,0}{{3.4}}}}
\fontsize{10}{0}
\selectfont\put(258.75,412.857){\makebox(0,0)[b]{\textcolor[rgb]{0,0,0}{{Method I: $g(x) = 1$}}}}
\fontsize{10}{0}
\selectfont\put(65,50.7221){\makebox(0,0)[t]{\textcolor[rgb]{0,0,0}{{1e+0}}}}
\fontsize{10}{0}
\selectfont\put(120.357,50.7221){\makebox(0,0)[t]{\textcolor[rgb]{0,0,0}{{1e+1}}}}
\fontsize{10}{0}
\selectfont\put(175.714,50.7221){\makebox(0,0)[t]{\textcolor[rgb]{0,0,0}{{1e+2}}}}
\fontsize{10}{0}
\selectfont\put(231.071,50.7221){\makebox(0,0)[t]{\textcolor[rgb]{0,0,0}{{1e+3}}}}
\fontsize{10}{0}
\selectfont\put(286.429,50.7221){\makebox(0,0)[t]{\textcolor[rgb]{0,0,0}{{1e+4}}}}
\fontsize{10}{0}
\selectfont\put(341.786,50.7221){\makebox(0,0)[t]{\textcolor[rgb]{0,0,0}{{1e+5}}}}
\fontsize{10}{0}
\selectfont\put(397.143,50.7221){\makebox(0,0)[t]{\textcolor[rgb]{0,0,0}{{1e+6}}}}
\fontsize{10}{0}
\selectfont\put(452.5,50.7221){\makebox(0,0)[t]{\textcolor[rgb]{0,0,0}{{1e+7}}}}
\fontsize{10}{0}
\selectfont\put(60.0065,55.7267){\makebox(0,0)[r]{\textcolor[rgb]{0,0,0}{{1e-4}}}}
\fontsize{10}{0}
\selectfont\put(60.0065,90.7593){\makebox(0,0)[r]{\textcolor[rgb]{0,0,0}{{1e-3}}}}
\fontsize{10}{0}
\selectfont\put(60.0065,125.792){\makebox(0,0)[r]{\textcolor[rgb]{0,0,0}{{1e-2}}}}
\fontsize{10}{0}
\selectfont\put(60.0065,160.825){\makebox(0,0)[r]{\textcolor[rgb]{0,0,0}{{1e-1}}}}
\fontsize{10}{0}
\selectfont\put(60.0065,195.857){\makebox(0,0)[r]{\textcolor[rgb]{0,0,0}{{1e+0}}}}
\fontsize{10}{0}
\selectfont\put(258.75,205.857){\makebox(0,0)[b]{\textcolor[rgb]{0,0,0}{{Method I}}}}
\fontsize{10}{0}
\selectfont\put(322.733,184.603){\makebox(0,0)[l]{\textcolor[rgb]{0,0,0}{{$\abs{err}$ vs number of trials}}}}
\fontsize{10}{0}
\selectfont\put(322.733,167.814){\makebox(0,0)[l]{\textcolor[rgb]{0,0,0}{{$\frac{1}{\sqrt{number of trials}}$}}}}
\end{picture}

	\endgroup
	\begingroup
		% Title: glps_renderer figure
% Creator: GL2PS 1.3.8, (C) 1999-2012 C. Geuzaine
% For: Octave
% CreationDate: Thu May  1 14:38:04 2014
\setlength{\unitlength}{1pt}
\begin{picture}(0,0)
\includegraphics{Method_II-inc}
\end{picture}%
\begin{picture}(500,450)(0,0)
\fontsize{10}{0}
\selectfont\put(65,257.722){\makebox(0,0)[t]{\textcolor[rgb]{0,0,0}{{1e+0}}}}
\fontsize{10}{0}
\selectfont\put(120.357,257.722){\makebox(0,0)[t]{\textcolor[rgb]{0,0,0}{{1e+1}}}}
\fontsize{10}{0}
\selectfont\put(175.714,257.722){\makebox(0,0)[t]{\textcolor[rgb]{0,0,0}{{1e+2}}}}
\fontsize{10}{0}
\selectfont\put(231.071,257.722){\makebox(0,0)[t]{\textcolor[rgb]{0,0,0}{{1e+3}}}}
\fontsize{10}{0}
\selectfont\put(286.429,257.722){\makebox(0,0)[t]{\textcolor[rgb]{0,0,0}{{1e+4}}}}
\fontsize{10}{0}
\selectfont\put(341.786,257.722){\makebox(0,0)[t]{\textcolor[rgb]{0,0,0}{{1e+5}}}}
\fontsize{10}{0}
\selectfont\put(397.143,257.722){\makebox(0,0)[t]{\textcolor[rgb]{0,0,0}{{1e+6}}}}
\fontsize{10}{0}
\selectfont\put(452.5,257.722){\makebox(0,0)[t]{\textcolor[rgb]{0,0,0}{{1e+7}}}}
\fontsize{10}{0}
\selectfont\put(60.0065,262.727){\makebox(0,0)[r]{\textcolor[rgb]{0,0,0}{{2.9}}}}
\fontsize{10}{0}
\selectfont\put(60.0065,290.753){\makebox(0,0)[r]{\textcolor[rgb]{0,0,0}{{3}}}}
\fontsize{10}{0}
\selectfont\put(60.0065,318.779){\makebox(0,0)[r]{\textcolor[rgb]{0,0,0}{{3.1}}}}
\fontsize{10}{0}
\selectfont\put(60.0065,346.805){\makebox(0,0)[r]{\textcolor[rgb]{0,0,0}{{3.2}}}}
\fontsize{10}{0}
\selectfont\put(60.0065,374.831){\makebox(0,0)[r]{\textcolor[rgb]{0,0,0}{{3.3}}}}
\fontsize{10}{0}
\selectfont\put(60.0065,402.857){\makebox(0,0)[r]{\textcolor[rgb]{0,0,0}{{3.4}}}}
\fontsize{10}{0}
\selectfont\put(258.75,412.857){\makebox(0,0)[b]{\textcolor[rgb]{0,0,0}{{Method II: $g(x) = \frac{4-2x}{3}$}}}}
\fontsize{10}{0}
\selectfont\put(65,50.7221){\makebox(0,0)[t]{\textcolor[rgb]{0,0,0}{{1e+0}}}}
\fontsize{10}{0}
\selectfont\put(120.357,50.7221){\makebox(0,0)[t]{\textcolor[rgb]{0,0,0}{{1e+1}}}}
\fontsize{10}{0}
\selectfont\put(175.714,50.7221){\makebox(0,0)[t]{\textcolor[rgb]{0,0,0}{{1e+2}}}}
\fontsize{10}{0}
\selectfont\put(231.071,50.7221){\makebox(0,0)[t]{\textcolor[rgb]{0,0,0}{{1e+3}}}}
\fontsize{10}{0}
\selectfont\put(286.429,50.7221){\makebox(0,0)[t]{\textcolor[rgb]{0,0,0}{{1e+4}}}}
\fontsize{10}{0}
\selectfont\put(341.786,50.7221){\makebox(0,0)[t]{\textcolor[rgb]{0,0,0}{{1e+5}}}}
\fontsize{10}{0}
\selectfont\put(397.143,50.7221){\makebox(0,0)[t]{\textcolor[rgb]{0,0,0}{{1e+6}}}}
\fontsize{10}{0}
\selectfont\put(452.5,50.7221){\makebox(0,0)[t]{\textcolor[rgb]{0,0,0}{{1e+7}}}}
\fontsize{10}{0}
\selectfont\put(60.0065,55.7267){\makebox(0,0)[r]{\textcolor[rgb]{0,0,0}{{1e-5}}}}
\fontsize{10}{0}
\selectfont\put(60.0065,83.7528){\makebox(0,0)[r]{\textcolor[rgb]{0,0,0}{{1e-4}}}}
\fontsize{10}{0}
\selectfont\put(60.0065,111.779){\makebox(0,0)[r]{\textcolor[rgb]{0,0,0}{{1e-3}}}}
\fontsize{10}{0}
\selectfont\put(60.0065,139.805){\makebox(0,0)[r]{\textcolor[rgb]{0,0,0}{{1e-2}}}}
\fontsize{10}{0}
\selectfont\put(60.0065,167.831){\makebox(0,0)[r]{\textcolor[rgb]{0,0,0}{{1e-1}}}}
\fontsize{10}{0}
\selectfont\put(60.0065,195.857){\makebox(0,0)[r]{\textcolor[rgb]{0,0,0}{{1e+0}}}}
\fontsize{10}{0}
\selectfont\put(258.75,205.857){\makebox(0,0)[b]{\textcolor[rgb]{0,0,0}{{Method II}}}}
\fontsize{10}{0}
\selectfont\put(322.733,184.603){\makebox(0,0)[l]{\textcolor[rgb]{0,0,0}{{$\abs{err}$ vs number of trials}}}}
\fontsize{10}{0}
\selectfont\put(322.733,167.814){\makebox(0,0)[l]{\textcolor[rgb]{0,0,0}{{$\frac{1}{\sqrt{number of trials}}$}}}}
\end{picture}

	\endgroup
	\begingroup
		% Title: glps_renderer figure
% Creator: GL2PS 1.3.8, (C) 1999-2012 C. Geuzaine
% For: Octave
% CreationDate: Thu May  1 14:38:05 2014
\setlength{\unitlength}{1pt}
\begin{picture}(0,0)
\includegraphics{Method_III-inc}
\end{picture}%
\begin{picture}(500,450)(0,0)
\fontsize{10}{0}
\selectfont\put(65,257.722){\makebox(0,0)[t]{\textcolor[rgb]{0,0,0}{{1e+0}}}}
\fontsize{10}{0}
\selectfont\put(120.357,257.722){\makebox(0,0)[t]{\textcolor[rgb]{0,0,0}{{1e+1}}}}
\fontsize{10}{0}
\selectfont\put(175.714,257.722){\makebox(0,0)[t]{\textcolor[rgb]{0,0,0}{{1e+2}}}}
\fontsize{10}{0}
\selectfont\put(231.071,257.722){\makebox(0,0)[t]{\textcolor[rgb]{0,0,0}{{1e+3}}}}
\fontsize{10}{0}
\selectfont\put(286.429,257.722){\makebox(0,0)[t]{\textcolor[rgb]{0,0,0}{{1e+4}}}}
\fontsize{10}{0}
\selectfont\put(341.786,257.722){\makebox(0,0)[t]{\textcolor[rgb]{0,0,0}{{1e+5}}}}
\fontsize{10}{0}
\selectfont\put(397.143,257.722){\makebox(0,0)[t]{\textcolor[rgb]{0,0,0}{{1e+6}}}}
\fontsize{10}{0}
\selectfont\put(452.5,257.722){\makebox(0,0)[t]{\textcolor[rgb]{0,0,0}{{1e+7}}}}
\fontsize{10}{0}
\selectfont\put(60.0065,262.727){\makebox(0,0)[r]{\textcolor[rgb]{0,0,0}{{2.5}}}}
\fontsize{10}{0}
\selectfont\put(60.0065,297.759){\makebox(0,0)[r]{\textcolor[rgb]{0,0,0}{{3}}}}
\fontsize{10}{0}
\selectfont\put(60.0065,332.792){\makebox(0,0)[r]{\textcolor[rgb]{0,0,0}{{3.5}}}}
\fontsize{10}{0}
\selectfont\put(60.0065,367.825){\makebox(0,0)[r]{\textcolor[rgb]{0,0,0}{{4}}}}
\fontsize{10}{0}
\selectfont\put(60.0065,402.857){\makebox(0,0)[r]{\textcolor[rgb]{0,0,0}{{4.5}}}}
\fontsize{10}{0}
\selectfont\put(258.75,412.857){\makebox(0,0)[b]{\textcolor[rgb]{0,0,0}{{Method III: $g(x) = 2-2x$}}}}
\fontsize{10}{0}
\selectfont\put(65,50.7221){\makebox(0,0)[t]{\textcolor[rgb]{0,0,0}{{1e+0}}}}
\fontsize{10}{0}
\selectfont\put(120.357,50.7221){\makebox(0,0)[t]{\textcolor[rgb]{0,0,0}{{1e+1}}}}
\fontsize{10}{0}
\selectfont\put(175.714,50.7221){\makebox(0,0)[t]{\textcolor[rgb]{0,0,0}{{1e+2}}}}
\fontsize{10}{0}
\selectfont\put(231.071,50.7221){\makebox(0,0)[t]{\textcolor[rgb]{0,0,0}{{1e+3}}}}
\fontsize{10}{0}
\selectfont\put(286.429,50.7221){\makebox(0,0)[t]{\textcolor[rgb]{0,0,0}{{1e+4}}}}
\fontsize{10}{0}
\selectfont\put(341.786,50.7221){\makebox(0,0)[t]{\textcolor[rgb]{0,0,0}{{1e+5}}}}
\fontsize{10}{0}
\selectfont\put(397.143,50.7221){\makebox(0,0)[t]{\textcolor[rgb]{0,0,0}{{1e+6}}}}
\fontsize{10}{0}
\selectfont\put(452.5,50.7221){\makebox(0,0)[t]{\textcolor[rgb]{0,0,0}{{1e+7}}}}
\fontsize{10}{0}
\selectfont\put(60.0065,55.7267){\makebox(0,0)[r]{\textcolor[rgb]{0,0,0}{{1e-4}}}}
\fontsize{10}{0}
\selectfont\put(60.0065,83.7528){\makebox(0,0)[r]{\textcolor[rgb]{0,0,0}{{1e-3}}}}
\fontsize{10}{0}
\selectfont\put(60.0065,111.779){\makebox(0,0)[r]{\textcolor[rgb]{0,0,0}{{1e-2}}}}
\fontsize{10}{0}
\selectfont\put(60.0065,139.805){\makebox(0,0)[r]{\textcolor[rgb]{0,0,0}{{1e-1}}}}
\fontsize{10}{0}
\selectfont\put(60.0065,167.831){\makebox(0,0)[r]{\textcolor[rgb]{0,0,0}{{1e+0}}}}
\fontsize{10}{0}
\selectfont\put(60.0065,195.857){\makebox(0,0)[r]{\textcolor[rgb]{0,0,0}{{1e+1}}}}
\fontsize{10}{0}
\selectfont\put(258.75,205.857){\makebox(0,0)[b]{\textcolor[rgb]{0,0,0}{{Method III}}}}
\fontsize{10}{0}
\selectfont\put(322.733,184.603){\makebox(0,0)[l]{\textcolor[rgb]{0,0,0}{{$\abs{err}$ vs number of trials}}}}
\fontsize{10}{0}
\selectfont\put(322.733,167.814){\makebox(0,0)[l]{\textcolor[rgb]{0,0,0}{{$\frac{1}{\sqrt{number of trials}}$}}}}
\end{picture}

	\endgroup
\end{homeworkProblem}

\clearpage

%----------------------------------------------------------------------------------------
%	PROBLEM 2
%----------------------------------------------------------------------------------------

\begin{homeworkProblem}
	\section{Code} 
	\begin{lstlisting}[language=Matlab]
clear; close all;
graphics_toolkit("gnuplot");

N_bins = 101;

function M = Metro_Ising(mu, N_steps, sample_rate, N)
	x = ones(N, 1);
	M = zeros(0, N_steps/sample_rate);
	for i = 1:N_steps
		j = ceil(N*rand);
		if j == 1
			h = exp(-2*mu*x(1)*(x(2)));
		elseif j == N
			h = exp(-2*mu*x(N)*(x(N-1)));
		else
			h = exp(-2*mu*x(j)*(x(j-1)+x(j+1)));
		end
		U = rand;
		if U <= h
			x(j) = -x(j);
		end
		if mod(i, 50) == 0
			M(i/50) = sum(x);
		end
	end
end

N_steps = 1e6;
sample_rate = 50;
N = 50;
%mesh = -N:floor(2*N/(N_bins - 1)):N;
%[n, h] = hist(M, mesh);
the_plot = figure();
M_h = Metro_Ising(1, N_steps, sample_rate, N);
M_l = Metro_Ising(2, N_steps, sample_rate, N);

h_plot = subplot(2,1,1);
hist(M_h, 101);
title(['Histogram of sum of states of high temperature ($\mu = 1$),
	'1D Ising model with $20,000$ from $1,000,000$ states']);
xlabel('Sum of states');
ylabel('Number of samples');

l_plot = subplot(2,1,2);
hist(M_l, 101);
title(['Histogram of sum of states of low temperature ($\mu = 2$),
	'1D Ising model with $20,000$ from $1,000,000$ states']);
xlabel('Sum of states');
ylabel('Number of samples');

print(the_plot, ['MetropolisIsing','.tex'],'-S520,400','-dtex')
	\end{lstlisting}  
	\section{Figures}
		\begingroup
			% GNUPLOT: LaTeX picture with Postscript
\begingroup
  \makeatletter
  \providecommand\color[2][]{%
    \GenericError{(gnuplot) \space\space\space\@spaces}{%
      Package color not loaded in conjunction with
      terminal option `colourtext'%
    }{See the gnuplot documentation for explanation.%
    }{Either use 'blacktext' in gnuplot or load the package
      color.sty in LaTeX.}%
    \renewcommand\color[2][]{}%
  }%
  \providecommand\includegraphics[2][]{%
    \GenericError{(gnuplot) \space\space\space\@spaces}{%
      Package graphicx or graphics not loaded%
    }{See the gnuplot documentation for explanation.%
    }{The gnuplot epslatex terminal needs graphicx.sty or graphics.sty.}%
    \renewcommand\includegraphics[2][]{}%
  }%
  \providecommand\rotatebox[2]{#2}%
  \@ifundefined{ifGPcolor}{%
    \newif\ifGPcolor
    \GPcolorfalse
  }{}%
  \@ifundefined{ifGPblacktext}{%
    \newif\ifGPblacktext
    \GPblacktexttrue
  }{}%
  % define a \g@addto@macro without @ in the name:
  \let\gplgaddtomacro\g@addto@macro
  % define empty templates for all commands taking text:
  \gdef\gplbacktext{}%
  \gdef\gplfronttext{}%
  \makeatother
  \ifGPblacktext
    % no textcolor at all
    \def\colorrgb#1{}%
    \def\colorgray#1{}%
  \else
    % gray or color?
    \ifGPcolor
      \def\colorrgb#1{\color[rgb]{#1}}%
      \def\colorgray#1{\color[gray]{#1}}%
      \expandafter\def\csname LTw\endcsname{\color{white}}%
      \expandafter\def\csname LTb\endcsname{\color{black}}%
      \expandafter\def\csname LTa\endcsname{\color{black}}%
      \expandafter\def\csname LT0\endcsname{\color[rgb]{1,0,0}}%
      \expandafter\def\csname LT1\endcsname{\color[rgb]{0,1,0}}%
      \expandafter\def\csname LT2\endcsname{\color[rgb]{0,0,1}}%
      \expandafter\def\csname LT3\endcsname{\color[rgb]{1,0,1}}%
      \expandafter\def\csname LT4\endcsname{\color[rgb]{0,1,1}}%
      \expandafter\def\csname LT5\endcsname{\color[rgb]{1,1,0}}%
      \expandafter\def\csname LT6\endcsname{\color[rgb]{0,0,0}}%
      \expandafter\def\csname LT7\endcsname{\color[rgb]{1,0.3,0}}%
      \expandafter\def\csname LT8\endcsname{\color[rgb]{0.5,0.5,0.5}}%
    \else
      % gray
      \def\colorrgb#1{\color{black}}%
      \def\colorgray#1{\color[gray]{#1}}%
      \expandafter\def\csname LTw\endcsname{\color{white}}%
      \expandafter\def\csname LTb\endcsname{\color{black}}%
      \expandafter\def\csname LTa\endcsname{\color{black}}%
      \expandafter\def\csname LT0\endcsname{\color{black}}%
      \expandafter\def\csname LT1\endcsname{\color{black}}%
      \expandafter\def\csname LT2\endcsname{\color{black}}%
      \expandafter\def\csname LT3\endcsname{\color{black}}%
      \expandafter\def\csname LT4\endcsname{\color{black}}%
      \expandafter\def\csname LT5\endcsname{\color{black}}%
      \expandafter\def\csname LT6\endcsname{\color{black}}%
      \expandafter\def\csname LT7\endcsname{\color{black}}%
      \expandafter\def\csname LT8\endcsname{\color{black}}%
    \fi
  \fi
  \setlength{\unitlength}{0.0500bp}%
  \begin{picture}(10400.00,8000.00)%
    \gplgaddtomacro\gplbacktext{%
      \colorrgb{0.00,0.00,0.00}%
      \put(1232,4670){\makebox(0,0)[r]{\strut{}0}}%
      \colorrgb{0.00,0.00,0.00}%
      \put(1232,5216){\makebox(0,0)[r]{\strut{}200}}%
      \colorrgb{0.00,0.00,0.00}%
      \put(1232,5762){\makebox(0,0)[r]{\strut{}400}}%
      \colorrgb{0.00,0.00,0.00}%
      \put(1232,6307){\makebox(0,0)[r]{\strut{}600}}%
      \colorrgb{0.00,0.00,0.00}%
      \put(1232,6853){\makebox(0,0)[r]{\strut{}800}}%
      \colorrgb{0.00,0.00,0.00}%
      \put(1232,7399){\makebox(0,0)[r]{\strut{}1000}}%
      \colorrgb{0.00,0.00,0.00}%
      \put(1352,4470){\makebox(0,0){\strut{}-60}}%
      \colorrgb{0.00,0.00,0.00}%
      \put(2695,4470){\makebox(0,0){\strut{}-40}}%
      \colorrgb{0.00,0.00,0.00}%
      \put(4038,4470){\makebox(0,0){\strut{}-20}}%
      \colorrgb{0.00,0.00,0.00}%
      \put(5382,4470){\makebox(0,0){\strut{}0}}%
      \colorrgb{0.00,0.00,0.00}%
      \put(6725,4470){\makebox(0,0){\strut{}20}}%
      \colorrgb{0.00,0.00,0.00}%
      \put(8068,4470){\makebox(0,0){\strut{}40}}%
      \colorrgb{0.00,0.00,0.00}%
      \put(9411,4470){\makebox(0,0){\strut{}60}}%
      \colorrgb{0.00,0.00,0.00}%
      \put(532,6034){\rotatebox{90}{\makebox(0,0){\strut{}Number of samples}}}%
      \colorrgb{0.00,0.00,0.00}%
      \put(5381,4170){\makebox(0,0){\strut{}Sum of states}}%
      \csname LTb\endcsname%
      \put(5381,7699){\makebox(0,0){\strut{}Histogram of sum of states of high temperature ($\mu = 1$) 1D Ising model with $20,000$ from $1,000,000$ states}}%
    }%
    \gplgaddtomacro\gplfronttext{%
    }%
    \gplgaddtomacro\gplbacktext{%
      \colorrgb{0.00,0.00,0.00}%
      \put(1232,880){\makebox(0,0)[r]{\strut{}0}}%
      \colorrgb{0.00,0.00,0.00}%
      \put(1232,1426){\makebox(0,0)[r]{\strut{}1000}}%
      \colorrgb{0.00,0.00,0.00}%
      \put(1232,1971){\makebox(0,0)[r]{\strut{}2000}}%
      \colorrgb{0.00,0.00,0.00}%
      \put(1232,2517){\makebox(0,0)[r]{\strut{}3000}}%
      \colorrgb{0.00,0.00,0.00}%
      \put(1232,3062){\makebox(0,0)[r]{\strut{}4000}}%
      \colorrgb{0.00,0.00,0.00}%
      \put(1232,3608){\makebox(0,0)[r]{\strut{}5000}}%
      \colorrgb{0.00,0.00,0.00}%
      \put(1352,680){\makebox(0,0){\strut{}-60}}%
      \colorrgb{0.00,0.00,0.00}%
      \put(2695,680){\makebox(0,0){\strut{}-40}}%
      \colorrgb{0.00,0.00,0.00}%
      \put(4038,680){\makebox(0,0){\strut{}-20}}%
      \colorrgb{0.00,0.00,0.00}%
      \put(5382,680){\makebox(0,0){\strut{}0}}%
      \colorrgb{0.00,0.00,0.00}%
      \put(6725,680){\makebox(0,0){\strut{}20}}%
      \colorrgb{0.00,0.00,0.00}%
      \put(8068,680){\makebox(0,0){\strut{}40}}%
      \colorrgb{0.00,0.00,0.00}%
      \put(9411,680){\makebox(0,0){\strut{}60}}%
      \colorrgb{0.00,0.00,0.00}%
      \put(532,2244){\rotatebox{90}{\makebox(0,0){\strut{}Number of samples}}}%
      \colorrgb{0.00,0.00,0.00}%
      \put(5381,380){\makebox(0,0){\strut{}Sum of states}}%
      \csname LTb\endcsname%
      \put(5381,3908){\makebox(0,0){\strut{}Histogram of sum of states of low temperature ($\mu = 2$) 1D Ising model with $20,000$ from $1,000,000$ states}}%
    }%
    \gplgaddtomacro\gplfronttext{%
    }%
    \gplbacktext
    \put(0,0){\includegraphics{MetropolisIsing}}%
    \gplfronttext
  \end{picture}%
\endgroup

		\endgroup
\end{homeworkProblem}

%----------------------------------------------------------------------------------------
\end{document}
