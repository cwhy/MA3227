%----------------------------------------------------------------------------------------
%	PACKAGES AND OTHER DOCUMENT CONFIGURATIONS
%----------------------------------------------------------------------------------------

\documentclass{article}

\usepackage{fancyhdr} % Required for custom headers
\usepackage{lastpage} % Required to determine the last page for the footer
\usepackage{extramarks} % Required for headers and footers
\usepackage[usenames,dvipsnames]{color} % Required for custom colors
\usepackage{graphicx} % Required to insert images
\usepackage{listings} % Required for insertion of code
\usepackage{courier} % Required for the courier font
\usepackage{amsmath,amsfonts,amsthm} % Math packages
\usepackage{enumitem}
% Margins
\topmargin=-0.45in
\evensidemargin=0in
\oddsidemargin=0in
\textwidth=6.5in
\textheight=9.0in
\headsep=0.25in

\linespread{1.1} % Line spacing

% Set up the header and footer
\pagestyle{fancy}
\lhead{\hmwkAuthorName} % Top left header
\rhead{\firstxmark} % Top right header
\lfoot{\lastxmark} % Bottom left footer
\cfoot{} % Bottom center footer
\rfoot{Page\ \thepage\ of\ 6} % Bottom right footer
\renewcommand\headrulewidth{0.4pt} % Size of the header rule
\renewcommand\footrulewidth{0.4pt} % Size of the footer rule

\setlength\parindent{0pt} % Removes all indentation from paragraphs
\providecommand{\abs}[1]{\lvert#1\rvert}

%----------------------------------------------------------------------------------------
%	DOCUMENT STRUCTURE COMMANDS
%	Skip this unless you know what you're doing
%----------------------------------------------------------------------------------------

% Header and footer for when a page split occurs within a problem environment
\newcommand{\enterProblemHeader}[1]{
	\nobreak\extramarks{#1}{#1 continued on next page\ldots}\nobreak
	\nobreak\extramarks{#1 (continued)}{#1 continued on next page\ldots}\nobreak
}

% Header and footer for when a page split occurs between problem environments
\newcommand{\exitProblemHeader}[1]{
	\nobreak\extramarks{#1 (continued)}{#1 continued on next page\ldots}\nobreak
	\nobreak\extramarks{#1}{}\nobreak
}

\setcounter{secnumdepth}{0} % Removes default section numbers
\newcounter{homeworkProblemCounter} % Creates a counter to keep track of the number of problems

\newcommand{\homeworkProblemName}{}
\newenvironment{homeworkProblem}[1][Part \roman{homeworkProblemCounter}]{ % Makes a new environment called homeworkProblem which takes 1 argument (custom name) but the default is "Problem #"
	\stepcounter{homeworkProblemCounter} % Increase counter for number of problems
	\renewcommand{\homeworkProblemName}{#1} % Assign \homeworkProblemName the name of the problem
	\section{\homeworkProblemName} % Make a section in the document with the custom problem count
	\enterProblemHeader{\homeworkProblemName} % Header and footer within the environment
}{
	\exitProblemHeader{\homeworkProblemName} % Header and footer after the environment
}

\newcommand{\problemAnswer}[1]{ % Defines the problem answer command with the content as the only argument
	\noindent\framebox[\columnwidth][c]{\begin{minipage}{0.98\columnwidth}#1\end{minipage}} % Makes the box around the problem answer and puts the content inside
}

\newcommand{\homeworkSectionName}{}
\newenvironment{homeworkSection}[1]{ % New environment for sections within homework problems, takes 1 argument - the name of the section
	\renewcommand{\homeworkSectionName}{#1} % Assign \homeworkSectionName to the name of the section from the environment argument
	\subsection{\homeworkSectionName} % Make a subsection with the custom name of the subsection
	\enterProblemHeader{\homeworkProblemName\ [\homeworkSectionName]} % Header and footer within the environment
}{
	\enterProblemHeader{\homeworkProblemName} % Header and footer after the environment
}

\lstloadlanguages{Matlab} % Load Perl syntax for listings, for a list of other languages supported see: ftp://ftp.tex.ac.uk/tex-archive/macros/latex/contrib/listings/listings.pdf
\lstset{language=Matlab, % Use Perl in this example
	frame=single, % Single frame around code
	basicstyle=\small\ttfamily, % Use small true type font
	keywordstyle=[1]\color{Blue}\bf, % Perl functions bold and blue
	keywordstyle=[2]\color{Purple}, % Perl function arguments purple
	keywordstyle=[3]\color{Blue}\underbar, % Custom functions underlined and blue
	identifierstyle=, % Nothing special about identifiers                                         
	commentstyle=\usefont{T1}{pcr}{m}{sl}\color{Green}\small, % Comments small dark green courier font
	stringstyle=\color{Purple}, % Strings are purple
	showstringspaces=false, % Don't put marks in string spaces
	tabsize=5, % 5 spaces per tab
	%
	% Put standard Perl functions not included in the default language here
	morekeywords={rand},
	%
	% Put Perl function parameters here
	morekeywords=[2]{on, off, interp},
	%
	% Put user defined functions here
	morekeywords=[3]{test},
	%
	morecomment=[l][\color{Blue}]{...}, % Line continuation (...) like blue comment
	numbers=left, % Line numbers on left
	firstnumber=1, % Line numbers start with line 1
	numberstyle=\tiny\color{Blue}, % Line numbers are blue and small
	stepnumber=5 % Line numbers go in steps of 5
}


%----------------------------------------------------------------------------------------
%	NAME AND CLASS SECTION
%----------------------------------------------------------------------------------------

\newcommand{\hmwkTitle}{Assignment\ \#3} % Assignment title
\newcommand{\hmwkDueDate}{Wednsday,\ 4\ Apr 2014} % Due date
\newcommand{\hmwkClass}{MA3227\ } % Course/class
\newcommand{\hmwkAuthorName}{Chen Yu}
\newcommand{\MatricNumber}{A0077976E}
\newcommand{\AuthorDescription}{Engineering Science Programme}

%----------------------------------------------------------------------------------------
%	TITLE PAGE
%----------------------------------------------------------------------------------------

\title{
	\vspace{2in}
	\textmd{\textbf{\hmwkClass:\ \hmwkTitle}}\\
	\normalsize\vspace{0.1in}\small{Due\ on\ \hmwkDueDate}
	\vspace{3in}
}

\author{
	\textbf{\hmwkAuthorName}\\
	\textbf{\MatricNumber}\\
	\textbf{\AuthorDescription}
}
\date{} % Insert date here if you want it to appear below your name

%----------------------------------------------------------------------------------------
\newcommand{\E}{\mathrm{E}}
\newcommand{\Var}{\mathrm{Var}}

\begin{document}

\maketitle

\newpage
\begin{homeworkProblem}[Homework Part I]
	\begin{homeworkSection}{(1)}
	  \problemAnswer{
		\textbf{Case 1:}
		$$\hat{V}_{B1} = \frac{\text{\#hits in B}}{\text{\#hits in A}}V_A$$
		Let $\hat\theta_1 = \frac{\text{\#hits in B}}{\text{\#hits in A}}$, which follows a binomial distribution with $N=n$, $p = \frac{V_B}{V_A}$\\
		Then $$\sigma_1 = \frac{\theta_1(1-\theta_1)}{n}V_A^2 = \frac{V_AV_B(V_A-V_B)}{n}$$
		\textbf{Case 2:}
		$$\hat{V}_{B2} = \frac{\text{\#hits in }B\setminus D}{\text{\#hits in }A\setminus D}(V_A-V_D) + V_{B\setminus D}$$
		Let $\hat\theta_2 = \frac{\text{\#hits in B\\D}}{\text{\#hits in} A\setminus D}$, which follows a binomial distribution with $N=n$, $p = \frac{V_{B\setminus D}}{V_{A\setminus D}}$\\
		Then
		\begin{align*} 
		  \sigma_2 &= \frac{\theta_2(1-\theta_2)}{n}(V_A-V_D)^2 
		  \\ &= \frac{ (V_A - V_D) V_{B\setminus D} (V_A-V_{B\setminus D}-V_D) }{n}
		  \\ &= \frac{(V_A-V_D)(V_B - V_{B\cap D})(V_A-V_{D\setminus B}-V_B)}{n}
		\end{align*}
		Since 
		\begin{align*} 
		  V_A &> V_A-V_D
		  \\ V_B &> V_B-V_{B_{B\cap D}}
		  \\ V_A-V_B &> V_A-V_B-V_{D\setminus B}
		\end{align*}
		we get $\sigma_1 > \sigma_2$
		}
	\end{homeworkSection}
	\begin{homeworkSection}{(2)}
	  \problemAnswer{
	  \textbf{a)}
	    In this method, the estimator follows a binomial distribution with $n =N$ and $p =\frac{I}{1} = I$\\
	    Hence the variance is 
	    \begin{align*}
	      \frac{1}{N}I(1-I) = \frac{1}{N}(I-I^2)
	    \end{align*}
	  \textbf{b)}
	    The variance of this Monte Carlo Integration is 
	    \begin{align*}
	     \Var [\frac{1}{N}\sum_{i=1}^n{f(U_i)}] &= \frac{1}{N^2}N\Var [f(U_i)]\\
	     &= \frac{1}{N} (\E[f(U_i)^2]-I^2)
	    \end{align*}
	    Since $$ \forall x \in [0,1], f(x) \in (0,1)$$
	    we have $$f(x)^2 < f(x)$$,
	    thus $$\E[f(U_i)^2] < \E[f(U_i)]$$
	    Therefore the method in b) has a smaller variance.
	    }
	\end{homeworkSection}
	\begin{homeworkSection}{(3)}
	  \problemAnswer{
	    \begin{align*}
	    (n-1)S_n^2 &= \sum_{i=1}^n{(X_i - \bar{X})^2}
	    \\ &= \sum_{i=1}^n{(X_i^2 - 2\bar{X}X_i+ \bar{x}^2)}
	    \\ &= \sum_{i=1}^n{X_i^2} - 2\bar{X}\sum_{i=1}^n{X_i}+ n\bar{x}^2
	    \\ &= \sum_{i=1}^n{X_i^2} - 2n\bar{X}^2 + n\bar{x}^2
	    \\ &= \sum_{i=1}^n{X_i^2} - n\bar{X}^2
	    \end{align*}
	    \begin{align*}
	    \E [S_n^2] &= \frac{1}{n-1} \E [ \sum_{i=1}^n{X_i^2} - n\bar{X}^2 ]
	    \\ &= \frac{1}{n-1} (n\E[X_i^2] - n\E[\bar{X}^2])
	    \\ &= \frac{n}{n-1} (\E[X_i^2] - \E[\bar{X}^2])
	    \end{align*}
	    \begin{align*}
	      \E [X_i^2] = (\E [X_i])^2 + \Var [X_i]
	    \end{align*}
	    \begin{align*}
	      \E [\bar{X}^2] &= (\E [\bar{X}])^2 + \Var [\bar{X}]&
	      \\ &= (\E[\frac{1}{n}\sum_{i=1}^n{X_i}])^2 + \Var [\frac{1}{n}\sum_{i=1}^n{X_i}]&
	      \\ &= (\E[X_i])^2 + \frac{1}{n} \Var [X_i] &\text{(i.i.d)}
	    \end{align*}
	    Therefore,
	    \begin{align*}
	      \E [S_n^2] &= \frac{n}{n-1} ((\E[X_i])^2 + \Var[X_i] - (\E[X_i])^2 -\frac{1}{n} \Var [X_i] )
	      \\ &= \Var[X_i]
	    \end{align*}
	    }
	\end{homeworkSection}
\end{homeworkProblem}
\clearpage

\begin{homeworkProblem}[Homework Part II(1)]
	  \problemAnswer{
	    Let $P_{x,y}$ be the transition probablity function. The transition probablity function generated by the new Metropolis algorithm is
	    $$
	    Q_{x,y} = \min \{ 1,\frac{P_{y,x}f(y)}{P_{x,y}f(x)}\}P_{x,y}
	    $$
	    \begin{align*}
	      f(x)Q_{x,y} &= f(x)\min \{ 1,\frac{P_{y,x}f(y)}{P_{x,y}f(x)}\}P_{x,y}
	      \\ &=  \min\{f(x)P_{y,x}, f(y)P_{y,x} \}
	    \end{align*}
	    \begin{align*}
	      f(y)Q_{y,x} &= f(y)\min \{1,\frac{P_{x,y}f(x)}{P_{y,x}f(y)} \}P_{y,x}
	      \\ &=  \min\{f(x)P_{y,x}, f(y)P_{y,x} \}
	    \end{align*}
	    \begin{align*}
	    \implies f(x)Q_{x,y} = f(y)Q_{y,x}
	    \end{align*}
	    , thus satisfying the detailed balance, which ensures that the Metropolis-Hasting chain generated has invariant distribution.
	  }
\end{homeworkProblem}
\clearpage

\begin{homeworkProblem}[Computer Project Part I]
	\begin{homeworkSection}{Code}
	\begin{itemize}
		\item[]
			\lstinputlisting{ass2_1.m}
	\end{itemize} 
	\end{homeworkSection}

	\begin{homeworkSection}{Figures}
		% Title: glps_renderer figure
% Creator: GL2PS 1.3.8, (C) 1999-2012 C. Geuzaine
% For: Octave
% CreationDate: Thu May  1 14:38:04 2014
\setlength{\unitlength}{1pt}
\begin{picture}(0,0)
\includegraphics{Method_I-inc}
\end{picture}%
\begin{picture}(500,450)(0,0)
\fontsize{10}{0}
\selectfont\put(65,257.722){\makebox(0,0)[t]{\textcolor[rgb]{0,0,0}{{1e+0}}}}
\fontsize{10}{0}
\selectfont\put(120.357,257.722){\makebox(0,0)[t]{\textcolor[rgb]{0,0,0}{{1e+1}}}}
\fontsize{10}{0}
\selectfont\put(175.714,257.722){\makebox(0,0)[t]{\textcolor[rgb]{0,0,0}{{1e+2}}}}
\fontsize{10}{0}
\selectfont\put(231.071,257.722){\makebox(0,0)[t]{\textcolor[rgb]{0,0,0}{{1e+3}}}}
\fontsize{10}{0}
\selectfont\put(286.429,257.722){\makebox(0,0)[t]{\textcolor[rgb]{0,0,0}{{1e+4}}}}
\fontsize{10}{0}
\selectfont\put(341.786,257.722){\makebox(0,0)[t]{\textcolor[rgb]{0,0,0}{{1e+5}}}}
\fontsize{10}{0}
\selectfont\put(397.143,257.722){\makebox(0,0)[t]{\textcolor[rgb]{0,0,0}{{1e+6}}}}
\fontsize{10}{0}
\selectfont\put(452.5,257.722){\makebox(0,0)[t]{\textcolor[rgb]{0,0,0}{{1e+7}}}}
\fontsize{10}{0}
\selectfont\put(60.0065,262.727){\makebox(0,0)[r]{\textcolor[rgb]{0,0,0}{{2.2}}}}
\fontsize{10}{0}
\selectfont\put(60.0065,286.082){\makebox(0,0)[r]{\textcolor[rgb]{0,0,0}{{2.4}}}}
\fontsize{10}{0}
\selectfont\put(60.0065,309.437){\makebox(0,0)[r]{\textcolor[rgb]{0,0,0}{{2.6}}}}
\fontsize{10}{0}
\selectfont\put(60.0065,332.792){\makebox(0,0)[r]{\textcolor[rgb]{0,0,0}{{2.8}}}}
\fontsize{10}{0}
\selectfont\put(60.0065,356.147){\makebox(0,0)[r]{\textcolor[rgb]{0,0,0}{{3}}}}
\fontsize{10}{0}
\selectfont\put(60.0065,379.502){\makebox(0,0)[r]{\textcolor[rgb]{0,0,0}{{3.2}}}}
\fontsize{10}{0}
\selectfont\put(60.0065,402.857){\makebox(0,0)[r]{\textcolor[rgb]{0,0,0}{{3.4}}}}
\fontsize{10}{0}
\selectfont\put(258.75,412.857){\makebox(0,0)[b]{\textcolor[rgb]{0,0,0}{{Method I: $g(x) = 1$}}}}
\fontsize{10}{0}
\selectfont\put(65,50.7221){\makebox(0,0)[t]{\textcolor[rgb]{0,0,0}{{1e+0}}}}
\fontsize{10}{0}
\selectfont\put(120.357,50.7221){\makebox(0,0)[t]{\textcolor[rgb]{0,0,0}{{1e+1}}}}
\fontsize{10}{0}
\selectfont\put(175.714,50.7221){\makebox(0,0)[t]{\textcolor[rgb]{0,0,0}{{1e+2}}}}
\fontsize{10}{0}
\selectfont\put(231.071,50.7221){\makebox(0,0)[t]{\textcolor[rgb]{0,0,0}{{1e+3}}}}
\fontsize{10}{0}
\selectfont\put(286.429,50.7221){\makebox(0,0)[t]{\textcolor[rgb]{0,0,0}{{1e+4}}}}
\fontsize{10}{0}
\selectfont\put(341.786,50.7221){\makebox(0,0)[t]{\textcolor[rgb]{0,0,0}{{1e+5}}}}
\fontsize{10}{0}
\selectfont\put(397.143,50.7221){\makebox(0,0)[t]{\textcolor[rgb]{0,0,0}{{1e+6}}}}
\fontsize{10}{0}
\selectfont\put(452.5,50.7221){\makebox(0,0)[t]{\textcolor[rgb]{0,0,0}{{1e+7}}}}
\fontsize{10}{0}
\selectfont\put(60.0065,55.7267){\makebox(0,0)[r]{\textcolor[rgb]{0,0,0}{{1e-4}}}}
\fontsize{10}{0}
\selectfont\put(60.0065,90.7593){\makebox(0,0)[r]{\textcolor[rgb]{0,0,0}{{1e-3}}}}
\fontsize{10}{0}
\selectfont\put(60.0065,125.792){\makebox(0,0)[r]{\textcolor[rgb]{0,0,0}{{1e-2}}}}
\fontsize{10}{0}
\selectfont\put(60.0065,160.825){\makebox(0,0)[r]{\textcolor[rgb]{0,0,0}{{1e-1}}}}
\fontsize{10}{0}
\selectfont\put(60.0065,195.857){\makebox(0,0)[r]{\textcolor[rgb]{0,0,0}{{1e+0}}}}
\fontsize{10}{0}
\selectfont\put(258.75,205.857){\makebox(0,0)[b]{\textcolor[rgb]{0,0,0}{{Method I}}}}
\fontsize{10}{0}
\selectfont\put(322.733,184.603){\makebox(0,0)[l]{\textcolor[rgb]{0,0,0}{{$\abs{err}$ vs number of trials}}}}
\fontsize{10}{0}
\selectfont\put(322.733,167.814){\makebox(0,0)[l]{\textcolor[rgb]{0,0,0}{{$\frac{1}{\sqrt{number of trials}}$}}}}
\end{picture}

		% Title: glps_renderer figure
% Creator: GL2PS 1.3.8, (C) 1999-2012 C. Geuzaine
% For: Octave
% CreationDate: Thu May  1 14:38:04 2014
\setlength{\unitlength}{1pt}
\begin{picture}(0,0)
\includegraphics{Method_II-inc}
\end{picture}%
\begin{picture}(500,450)(0,0)
\fontsize{10}{0}
\selectfont\put(65,257.722){\makebox(0,0)[t]{\textcolor[rgb]{0,0,0}{{1e+0}}}}
\fontsize{10}{0}
\selectfont\put(120.357,257.722){\makebox(0,0)[t]{\textcolor[rgb]{0,0,0}{{1e+1}}}}
\fontsize{10}{0}
\selectfont\put(175.714,257.722){\makebox(0,0)[t]{\textcolor[rgb]{0,0,0}{{1e+2}}}}
\fontsize{10}{0}
\selectfont\put(231.071,257.722){\makebox(0,0)[t]{\textcolor[rgb]{0,0,0}{{1e+3}}}}
\fontsize{10}{0}
\selectfont\put(286.429,257.722){\makebox(0,0)[t]{\textcolor[rgb]{0,0,0}{{1e+4}}}}
\fontsize{10}{0}
\selectfont\put(341.786,257.722){\makebox(0,0)[t]{\textcolor[rgb]{0,0,0}{{1e+5}}}}
\fontsize{10}{0}
\selectfont\put(397.143,257.722){\makebox(0,0)[t]{\textcolor[rgb]{0,0,0}{{1e+6}}}}
\fontsize{10}{0}
\selectfont\put(452.5,257.722){\makebox(0,0)[t]{\textcolor[rgb]{0,0,0}{{1e+7}}}}
\fontsize{10}{0}
\selectfont\put(60.0065,262.727){\makebox(0,0)[r]{\textcolor[rgb]{0,0,0}{{2.9}}}}
\fontsize{10}{0}
\selectfont\put(60.0065,290.753){\makebox(0,0)[r]{\textcolor[rgb]{0,0,0}{{3}}}}
\fontsize{10}{0}
\selectfont\put(60.0065,318.779){\makebox(0,0)[r]{\textcolor[rgb]{0,0,0}{{3.1}}}}
\fontsize{10}{0}
\selectfont\put(60.0065,346.805){\makebox(0,0)[r]{\textcolor[rgb]{0,0,0}{{3.2}}}}
\fontsize{10}{0}
\selectfont\put(60.0065,374.831){\makebox(0,0)[r]{\textcolor[rgb]{0,0,0}{{3.3}}}}
\fontsize{10}{0}
\selectfont\put(60.0065,402.857){\makebox(0,0)[r]{\textcolor[rgb]{0,0,0}{{3.4}}}}
\fontsize{10}{0}
\selectfont\put(258.75,412.857){\makebox(0,0)[b]{\textcolor[rgb]{0,0,0}{{Method II: $g(x) = \frac{4-2x}{3}$}}}}
\fontsize{10}{0}
\selectfont\put(65,50.7221){\makebox(0,0)[t]{\textcolor[rgb]{0,0,0}{{1e+0}}}}
\fontsize{10}{0}
\selectfont\put(120.357,50.7221){\makebox(0,0)[t]{\textcolor[rgb]{0,0,0}{{1e+1}}}}
\fontsize{10}{0}
\selectfont\put(175.714,50.7221){\makebox(0,0)[t]{\textcolor[rgb]{0,0,0}{{1e+2}}}}
\fontsize{10}{0}
\selectfont\put(231.071,50.7221){\makebox(0,0)[t]{\textcolor[rgb]{0,0,0}{{1e+3}}}}
\fontsize{10}{0}
\selectfont\put(286.429,50.7221){\makebox(0,0)[t]{\textcolor[rgb]{0,0,0}{{1e+4}}}}
\fontsize{10}{0}
\selectfont\put(341.786,50.7221){\makebox(0,0)[t]{\textcolor[rgb]{0,0,0}{{1e+5}}}}
\fontsize{10}{0}
\selectfont\put(397.143,50.7221){\makebox(0,0)[t]{\textcolor[rgb]{0,0,0}{{1e+6}}}}
\fontsize{10}{0}
\selectfont\put(452.5,50.7221){\makebox(0,0)[t]{\textcolor[rgb]{0,0,0}{{1e+7}}}}
\fontsize{10}{0}
\selectfont\put(60.0065,55.7267){\makebox(0,0)[r]{\textcolor[rgb]{0,0,0}{{1e-5}}}}
\fontsize{10}{0}
\selectfont\put(60.0065,83.7528){\makebox(0,0)[r]{\textcolor[rgb]{0,0,0}{{1e-4}}}}
\fontsize{10}{0}
\selectfont\put(60.0065,111.779){\makebox(0,0)[r]{\textcolor[rgb]{0,0,0}{{1e-3}}}}
\fontsize{10}{0}
\selectfont\put(60.0065,139.805){\makebox(0,0)[r]{\textcolor[rgb]{0,0,0}{{1e-2}}}}
\fontsize{10}{0}
\selectfont\put(60.0065,167.831){\makebox(0,0)[r]{\textcolor[rgb]{0,0,0}{{1e-1}}}}
\fontsize{10}{0}
\selectfont\put(60.0065,195.857){\makebox(0,0)[r]{\textcolor[rgb]{0,0,0}{{1e+0}}}}
\fontsize{10}{0}
\selectfont\put(258.75,205.857){\makebox(0,0)[b]{\textcolor[rgb]{0,0,0}{{Method II}}}}
\fontsize{10}{0}
\selectfont\put(322.733,184.603){\makebox(0,0)[l]{\textcolor[rgb]{0,0,0}{{$\abs{err}$ vs number of trials}}}}
\fontsize{10}{0}
\selectfont\put(322.733,167.814){\makebox(0,0)[l]{\textcolor[rgb]{0,0,0}{{$\frac{1}{\sqrt{number of trials}}$}}}}
\end{picture}

		% Title: glps_renderer figure
% Creator: GL2PS 1.3.8, (C) 1999-2012 C. Geuzaine
% For: Octave
% CreationDate: Thu May  1 14:38:05 2014
\setlength{\unitlength}{1pt}
\begin{picture}(0,0)
\includegraphics{Method_III-inc}
\end{picture}%
\begin{picture}(500,450)(0,0)
\fontsize{10}{0}
\selectfont\put(65,257.722){\makebox(0,0)[t]{\textcolor[rgb]{0,0,0}{{1e+0}}}}
\fontsize{10}{0}
\selectfont\put(120.357,257.722){\makebox(0,0)[t]{\textcolor[rgb]{0,0,0}{{1e+1}}}}
\fontsize{10}{0}
\selectfont\put(175.714,257.722){\makebox(0,0)[t]{\textcolor[rgb]{0,0,0}{{1e+2}}}}
\fontsize{10}{0}
\selectfont\put(231.071,257.722){\makebox(0,0)[t]{\textcolor[rgb]{0,0,0}{{1e+3}}}}
\fontsize{10}{0}
\selectfont\put(286.429,257.722){\makebox(0,0)[t]{\textcolor[rgb]{0,0,0}{{1e+4}}}}
\fontsize{10}{0}
\selectfont\put(341.786,257.722){\makebox(0,0)[t]{\textcolor[rgb]{0,0,0}{{1e+5}}}}
\fontsize{10}{0}
\selectfont\put(397.143,257.722){\makebox(0,0)[t]{\textcolor[rgb]{0,0,0}{{1e+6}}}}
\fontsize{10}{0}
\selectfont\put(452.5,257.722){\makebox(0,0)[t]{\textcolor[rgb]{0,0,0}{{1e+7}}}}
\fontsize{10}{0}
\selectfont\put(60.0065,262.727){\makebox(0,0)[r]{\textcolor[rgb]{0,0,0}{{2.5}}}}
\fontsize{10}{0}
\selectfont\put(60.0065,297.759){\makebox(0,0)[r]{\textcolor[rgb]{0,0,0}{{3}}}}
\fontsize{10}{0}
\selectfont\put(60.0065,332.792){\makebox(0,0)[r]{\textcolor[rgb]{0,0,0}{{3.5}}}}
\fontsize{10}{0}
\selectfont\put(60.0065,367.825){\makebox(0,0)[r]{\textcolor[rgb]{0,0,0}{{4}}}}
\fontsize{10}{0}
\selectfont\put(60.0065,402.857){\makebox(0,0)[r]{\textcolor[rgb]{0,0,0}{{4.5}}}}
\fontsize{10}{0}
\selectfont\put(258.75,412.857){\makebox(0,0)[b]{\textcolor[rgb]{0,0,0}{{Method III: $g(x) = 2-2x$}}}}
\fontsize{10}{0}
\selectfont\put(65,50.7221){\makebox(0,0)[t]{\textcolor[rgb]{0,0,0}{{1e+0}}}}
\fontsize{10}{0}
\selectfont\put(120.357,50.7221){\makebox(0,0)[t]{\textcolor[rgb]{0,0,0}{{1e+1}}}}
\fontsize{10}{0}
\selectfont\put(175.714,50.7221){\makebox(0,0)[t]{\textcolor[rgb]{0,0,0}{{1e+2}}}}
\fontsize{10}{0}
\selectfont\put(231.071,50.7221){\makebox(0,0)[t]{\textcolor[rgb]{0,0,0}{{1e+3}}}}
\fontsize{10}{0}
\selectfont\put(286.429,50.7221){\makebox(0,0)[t]{\textcolor[rgb]{0,0,0}{{1e+4}}}}
\fontsize{10}{0}
\selectfont\put(341.786,50.7221){\makebox(0,0)[t]{\textcolor[rgb]{0,0,0}{{1e+5}}}}
\fontsize{10}{0}
\selectfont\put(397.143,50.7221){\makebox(0,0)[t]{\textcolor[rgb]{0,0,0}{{1e+6}}}}
\fontsize{10}{0}
\selectfont\put(452.5,50.7221){\makebox(0,0)[t]{\textcolor[rgb]{0,0,0}{{1e+7}}}}
\fontsize{10}{0}
\selectfont\put(60.0065,55.7267){\makebox(0,0)[r]{\textcolor[rgb]{0,0,0}{{1e-4}}}}
\fontsize{10}{0}
\selectfont\put(60.0065,83.7528){\makebox(0,0)[r]{\textcolor[rgb]{0,0,0}{{1e-3}}}}
\fontsize{10}{0}
\selectfont\put(60.0065,111.779){\makebox(0,0)[r]{\textcolor[rgb]{0,0,0}{{1e-2}}}}
\fontsize{10}{0}
\selectfont\put(60.0065,139.805){\makebox(0,0)[r]{\textcolor[rgb]{0,0,0}{{1e-1}}}}
\fontsize{10}{0}
\selectfont\put(60.0065,167.831){\makebox(0,0)[r]{\textcolor[rgb]{0,0,0}{{1e+0}}}}
\fontsize{10}{0}
\selectfont\put(60.0065,195.857){\makebox(0,0)[r]{\textcolor[rgb]{0,0,0}{{1e+1}}}}
\fontsize{10}{0}
\selectfont\put(258.75,205.857){\makebox(0,0)[b]{\textcolor[rgb]{0,0,0}{{Method III}}}}
\fontsize{10}{0}
\selectfont\put(322.733,184.603){\makebox(0,0)[l]{\textcolor[rgb]{0,0,0}{{$\abs{err}$ vs number of trials}}}}
\fontsize{10}{0}
\selectfont\put(322.733,167.814){\makebox(0,0)[l]{\textcolor[rgb]{0,0,0}{{$\frac{1}{\sqrt{number of trials}}$}}}}
\end{picture}

		% GNUPLOT: LaTeX picture with Postscript
\begingroup
  \makeatletter
  \providecommand\color[2][]{%
    \GenericError{(gnuplot) \space\space\space\@spaces}{%
      Package color not loaded in conjunction with
      terminal option `colourtext'%
    }{See the gnuplot documentation for explanation.%
    }{Either use 'blacktext' in gnuplot or load the package
      color.sty in LaTeX.}%
    \renewcommand\color[2][]{}%
  }%
  \providecommand\includegraphics[2][]{%
    \GenericError{(gnuplot) \space\space\space\@spaces}{%
      Package graphicx or graphics not loaded%
    }{See the gnuplot documentation for explanation.%
    }{The gnuplot epslatex terminal needs graphicx.sty or graphics.sty.}%
    \renewcommand\includegraphics[2][]{}%
  }%
  \providecommand\rotatebox[2]{#2}%
  \@ifundefined{ifGPcolor}{%
    \newif\ifGPcolor
    \GPcolorfalse
  }{}%
  \@ifundefined{ifGPblacktext}{%
    \newif\ifGPblacktext
    \GPblacktexttrue
  }{}%
  % define a \g@addto@macro without @ in the name:
  \let\gplgaddtomacro\g@addto@macro
  % define empty templates for all commands taking text:
  \gdef\gplbacktext{}%
  \gdef\gplfronttext{}%
  \makeatother
  \ifGPblacktext
    % no textcolor at all
    \def\colorrgb#1{}%
    \def\colorgray#1{}%
  \else
    % gray or color?
    \ifGPcolor
      \def\colorrgb#1{\color[rgb]{#1}}%
      \def\colorgray#1{\color[gray]{#1}}%
      \expandafter\def\csname LTw\endcsname{\color{white}}%
      \expandafter\def\csname LTb\endcsname{\color{black}}%
      \expandafter\def\csname LTa\endcsname{\color{black}}%
      \expandafter\def\csname LT0\endcsname{\color[rgb]{1,0,0}}%
      \expandafter\def\csname LT1\endcsname{\color[rgb]{0,1,0}}%
      \expandafter\def\csname LT2\endcsname{\color[rgb]{0,0,1}}%
      \expandafter\def\csname LT3\endcsname{\color[rgb]{1,0,1}}%
      \expandafter\def\csname LT4\endcsname{\color[rgb]{0,1,1}}%
      \expandafter\def\csname LT5\endcsname{\color[rgb]{1,1,0}}%
      \expandafter\def\csname LT6\endcsname{\color[rgb]{0,0,0}}%
      \expandafter\def\csname LT7\endcsname{\color[rgb]{1,0.3,0}}%
      \expandafter\def\csname LT8\endcsname{\color[rgb]{0.5,0.5,0.5}}%
    \else
      % gray
      \def\colorrgb#1{\color{black}}%
      \def\colorgray#1{\color[gray]{#1}}%
      \expandafter\def\csname LTw\endcsname{\color{white}}%
      \expandafter\def\csname LTb\endcsname{\color{black}}%
      \expandafter\def\csname LTa\endcsname{\color{black}}%
      \expandafter\def\csname LT0\endcsname{\color{black}}%
      \expandafter\def\csname LT1\endcsname{\color{black}}%
      \expandafter\def\csname LT2\endcsname{\color{black}}%
      \expandafter\def\csname LT3\endcsname{\color{black}}%
      \expandafter\def\csname LT4\endcsname{\color{black}}%
      \expandafter\def\csname LT5\endcsname{\color{black}}%
      \expandafter\def\csname LT6\endcsname{\color{black}}%
      \expandafter\def\csname LT7\endcsname{\color{black}}%
      \expandafter\def\csname LT8\endcsname{\color{black}}%
    \fi
  \fi
  \setlength{\unitlength}{0.0500bp}%
  \begin{picture}(9400.00,3000.00)%
    \gplgaddtomacro\gplbacktext{%
      \colorrgb{0.00,0.00,0.00}%
      \put(1740,733){\makebox(0,0)[r]{\strut{}$g(x) = 1$}}%
      \colorrgb{0.00,0.00,0.00}%
      \put(1740,1400){\makebox(0,0)[r]{\strut{}$g(x) = \frac{4-2x}{3}$}}%
      \colorrgb{0.00,0.00,0.00}%
      \put(1740,2066){\makebox(0,0)[r]{\strut{}$g(x) = 2-2x$}}%
      \colorrgb{0.00,0.00,0.00}%
      \put(1860,200){\makebox(0,0){\strut{}0}}%
      \colorrgb{0.00,0.00,0.00}%
      \put(3296,200){\makebox(0,0){\strut{}1}}%
      \colorrgb{0.00,0.00,0.00}%
      \put(4732,200){\makebox(0,0){\strut{}2}}%
      \colorrgb{0.00,0.00,0.00}%
      \put(6167,200){\makebox(0,0){\strut{}3}}%
      \colorrgb{0.00,0.00,0.00}%
      \put(7603,200){\makebox(0,0){\strut{}4}}%
      \colorrgb{0.00,0.00,0.00}%
      \put(9039,200){\makebox(0,0){\strut{}5}}%
      \csname LTb\endcsname%
      \put(5449,2699){\makebox(0,0){\strut{}$\abs{Err} \times \sqrt{\text{number of trials}}$ for different methods}}%
    }%
    \gplgaddtomacro\gplfronttext{%
    }%
    \gplbacktext
    \put(0,0){\includegraphics{comparison}}%
    \gplfronttext
  \end{picture}%
\endgroup

	\end{homeworkSection}
\end{homeworkProblem}

\clearpage

\begin{homeworkProblem}[Computer Project Part II]
	\begin{homeworkSection}{Code}
	\begin{itemize}
		\item[]
			\lstinputlisting{ass2_2.m}
	\end{itemize} 
	\end{homeworkSection}

	\begin{homeworkSection}{Figures}
		% GNUPLOT: LaTeX picture with Postscript
\begingroup
  \makeatletter
  \providecommand\color[2][]{%
    \GenericError{(gnuplot) \space\space\space\@spaces}{%
      Package color not loaded in conjunction with
      terminal option `colourtext'%
    }{See the gnuplot documentation for explanation.%
    }{Either use 'blacktext' in gnuplot or load the package
      color.sty in LaTeX.}%
    \renewcommand\color[2][]{}%
  }%
  \providecommand\includegraphics[2][]{%
    \GenericError{(gnuplot) \space\space\space\@spaces}{%
      Package graphicx or graphics not loaded%
    }{See the gnuplot documentation for explanation.%
    }{The gnuplot epslatex terminal needs graphicx.sty or graphics.sty.}%
    \renewcommand\includegraphics[2][]{}%
  }%
  \providecommand\rotatebox[2]{#2}%
  \@ifundefined{ifGPcolor}{%
    \newif\ifGPcolor
    \GPcolorfalse
  }{}%
  \@ifundefined{ifGPblacktext}{%
    \newif\ifGPblacktext
    \GPblacktexttrue
  }{}%
  % define a \g@addto@macro without @ in the name:
  \let\gplgaddtomacro\g@addto@macro
  % define empty templates for all commands taking text:
  \gdef\gplbacktext{}%
  \gdef\gplfronttext{}%
  \makeatother
  \ifGPblacktext
    % no textcolor at all
    \def\colorrgb#1{}%
    \def\colorgray#1{}%
  \else
    % gray or color?
    \ifGPcolor
      \def\colorrgb#1{\color[rgb]{#1}}%
      \def\colorgray#1{\color[gray]{#1}}%
      \expandafter\def\csname LTw\endcsname{\color{white}}%
      \expandafter\def\csname LTb\endcsname{\color{black}}%
      \expandafter\def\csname LTa\endcsname{\color{black}}%
      \expandafter\def\csname LT0\endcsname{\color[rgb]{1,0,0}}%
      \expandafter\def\csname LT1\endcsname{\color[rgb]{0,1,0}}%
      \expandafter\def\csname LT2\endcsname{\color[rgb]{0,0,1}}%
      \expandafter\def\csname LT3\endcsname{\color[rgb]{1,0,1}}%
      \expandafter\def\csname LT4\endcsname{\color[rgb]{0,1,1}}%
      \expandafter\def\csname LT5\endcsname{\color[rgb]{1,1,0}}%
      \expandafter\def\csname LT6\endcsname{\color[rgb]{0,0,0}}%
      \expandafter\def\csname LT7\endcsname{\color[rgb]{1,0.3,0}}%
      \expandafter\def\csname LT8\endcsname{\color[rgb]{0.5,0.5,0.5}}%
    \else
      % gray
      \def\colorrgb#1{\color{black}}%
      \def\colorgray#1{\color[gray]{#1}}%
      \expandafter\def\csname LTw\endcsname{\color{white}}%
      \expandafter\def\csname LTb\endcsname{\color{black}}%
      \expandafter\def\csname LTa\endcsname{\color{black}}%
      \expandafter\def\csname LT0\endcsname{\color{black}}%
      \expandafter\def\csname LT1\endcsname{\color{black}}%
      \expandafter\def\csname LT2\endcsname{\color{black}}%
      \expandafter\def\csname LT3\endcsname{\color{black}}%
      \expandafter\def\csname LT4\endcsname{\color{black}}%
      \expandafter\def\csname LT5\endcsname{\color{black}}%
      \expandafter\def\csname LT6\endcsname{\color{black}}%
      \expandafter\def\csname LT7\endcsname{\color{black}}%
      \expandafter\def\csname LT8\endcsname{\color{black}}%
    \fi
  \fi
  \setlength{\unitlength}{0.0500bp}%
  \begin{picture}(10400.00,8000.00)%
    \gplgaddtomacro\gplbacktext{%
      \colorrgb{0.00,0.00,0.00}%
      \put(1232,4670){\makebox(0,0)[r]{\strut{}0}}%
      \colorrgb{0.00,0.00,0.00}%
      \put(1232,5216){\makebox(0,0)[r]{\strut{}200}}%
      \colorrgb{0.00,0.00,0.00}%
      \put(1232,5762){\makebox(0,0)[r]{\strut{}400}}%
      \colorrgb{0.00,0.00,0.00}%
      \put(1232,6307){\makebox(0,0)[r]{\strut{}600}}%
      \colorrgb{0.00,0.00,0.00}%
      \put(1232,6853){\makebox(0,0)[r]{\strut{}800}}%
      \colorrgb{0.00,0.00,0.00}%
      \put(1232,7399){\makebox(0,0)[r]{\strut{}1000}}%
      \colorrgb{0.00,0.00,0.00}%
      \put(1352,4470){\makebox(0,0){\strut{}-60}}%
      \colorrgb{0.00,0.00,0.00}%
      \put(2695,4470){\makebox(0,0){\strut{}-40}}%
      \colorrgb{0.00,0.00,0.00}%
      \put(4038,4470){\makebox(0,0){\strut{}-20}}%
      \colorrgb{0.00,0.00,0.00}%
      \put(5382,4470){\makebox(0,0){\strut{}0}}%
      \colorrgb{0.00,0.00,0.00}%
      \put(6725,4470){\makebox(0,0){\strut{}20}}%
      \colorrgb{0.00,0.00,0.00}%
      \put(8068,4470){\makebox(0,0){\strut{}40}}%
      \colorrgb{0.00,0.00,0.00}%
      \put(9411,4470){\makebox(0,0){\strut{}60}}%
      \colorrgb{0.00,0.00,0.00}%
      \put(532,6034){\rotatebox{90}{\makebox(0,0){\strut{}Number of samples}}}%
      \colorrgb{0.00,0.00,0.00}%
      \put(5381,4170){\makebox(0,0){\strut{}Sum of states}}%
      \csname LTb\endcsname%
      \put(5381,7699){\makebox(0,0){\strut{}Histogram of sum of states of high temperature ($\mu = 1$) 1D Ising model with $20,000$ from $1,000,000$ states}}%
    }%
    \gplgaddtomacro\gplfronttext{%
    }%
    \gplgaddtomacro\gplbacktext{%
      \colorrgb{0.00,0.00,0.00}%
      \put(1232,880){\makebox(0,0)[r]{\strut{}0}}%
      \colorrgb{0.00,0.00,0.00}%
      \put(1232,1426){\makebox(0,0)[r]{\strut{}1000}}%
      \colorrgb{0.00,0.00,0.00}%
      \put(1232,1971){\makebox(0,0)[r]{\strut{}2000}}%
      \colorrgb{0.00,0.00,0.00}%
      \put(1232,2517){\makebox(0,0)[r]{\strut{}3000}}%
      \colorrgb{0.00,0.00,0.00}%
      \put(1232,3062){\makebox(0,0)[r]{\strut{}4000}}%
      \colorrgb{0.00,0.00,0.00}%
      \put(1232,3608){\makebox(0,0)[r]{\strut{}5000}}%
      \colorrgb{0.00,0.00,0.00}%
      \put(1352,680){\makebox(0,0){\strut{}-60}}%
      \colorrgb{0.00,0.00,0.00}%
      \put(2695,680){\makebox(0,0){\strut{}-40}}%
      \colorrgb{0.00,0.00,0.00}%
      \put(4038,680){\makebox(0,0){\strut{}-20}}%
      \colorrgb{0.00,0.00,0.00}%
      \put(5382,680){\makebox(0,0){\strut{}0}}%
      \colorrgb{0.00,0.00,0.00}%
      \put(6725,680){\makebox(0,0){\strut{}20}}%
      \colorrgb{0.00,0.00,0.00}%
      \put(8068,680){\makebox(0,0){\strut{}40}}%
      \colorrgb{0.00,0.00,0.00}%
      \put(9411,680){\makebox(0,0){\strut{}60}}%
      \colorrgb{0.00,0.00,0.00}%
      \put(532,2244){\rotatebox{90}{\makebox(0,0){\strut{}Number of samples}}}%
      \colorrgb{0.00,0.00,0.00}%
      \put(5381,380){\makebox(0,0){\strut{}Sum of states}}%
      \csname LTb\endcsname%
      \put(5381,3908){\makebox(0,0){\strut{}Histogram of sum of states of low temperature ($\mu = 2$) 1D Ising model with $20,000$ from $1,000,000$ states}}%
    }%
    \gplgaddtomacro\gplfronttext{%
    }%
    \gplbacktext
    \put(0,0){\includegraphics{MetropolisIsing}}%
    \gplfronttext
  \end{picture}%
\endgroup

	\end{homeworkSection}
\end{homeworkProblem}

\begin{homeworkProblem}[Homework Part II(2)]
	\problemAnswer{
	  $f$ is the distribution of all microstates in the state space of $2^N$ possible micro states of $(x_1, x_2, ..., x_N)$, where $x_i = -1 \text{ or } 1$.\\
	  $P_{x,y} = \frac{1}{N}$ if $x_i = y_i$ for all but one component; which results in a proposed state $y$ where only the spin of one particle is reversed.
	}
\end{homeworkProblem}
%----------------------------------------------------------------------------------------
\end{document}
